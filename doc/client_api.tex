%%%%%%%%%%%%%%%%%%%%%%%%%%%%%%%%%%%%%%%%
% Copyright (c) IBM Corp. 2014
% All rights reserved. This program and the accompanying materials
% are made available under the terms of the Eclipse Public License v1.0
% which accompanies this distribution, and is available at
% http://www.eclipse.org/legal/epl-v10.html
%
% Contributors:
%    lschneid
%%%%%%%%%%%%%%%%%%%%%%%%%%%%%%%%%%%%%%%%

\section{Client Interface}\label{sec:api}
This section explains the client API in the way that applications
would use to access the SKV Server.

\subsection{Client Initialization and Exit}\label{sec:api:init}
\begin{lstlisting}
skv_status_t Init( skv_client_group_id_t aCommGroupId,
                   MPI_Comm aComm,
                   int aFlags,
                   const char* aConfigFile = NULL );
\end{lstlisting}

The first thing for every client to do is to initialize SKV.
\begin{description}
\item[aCommGroupId] This is an identifier for this group of clients.
  It helps the server to distinguish between different client groups
  and make sure data is retrieved or returned to the correct client.
  This setting is more important for multi-process client (\abrEG MPI
  based applications).
\item[aComm] The MPI communicator has to be provided for the MPI
  version of the client API (libskv\_mpi, libskv\_client\_mpi).
\item[aFlags] Initialization flags for SKV. Currently unused.
\item[aConfigFile] The path and name of the skv configuration file.
  This file contains several important settings for the server and the
  client.  The parameter is optional.  If omitted,
  \code{/etc/skv\_server.conf} will be used as a default.  See
  \ref{sec:skv:config} for details.
\end{description}
The initialization with this call can be thought of the client side
initialization of SKV.  It does not attempt to connect to any server.

\begin{lstlisting}
skv_status_t Finalize();
\end{lstlisting}

Finalize will cleanup any initialized and accumulated state of the SKV
client.


\subsection{Server Connection}\label{sec:api:connect}

\begin{lstlisting}
skv_status_t Connect( const char* aConfigFile,
                      int aFlags );
\end{lstlisting}

Before a client can access a server, it has to connect to a server.
\begin{description}
\item[aConfigFile] Since the server may consist of a group of
  processes, it's not meaningful to connect to a particular address.
  Instead, the config file contains the name and the path of the
  server's machinefile which contains a list of addresses and port
  numbers for the client to connect to.  This might change in the
  future, especially, since the config file was already provided with
  the Init() call (see Section\,\ref{sec:api:init}).
\item[aFlags] Connection flags for SKV. Currently unused.
\end{description}

NOTE: The parameters for this are under review.  Especially, the
config file parameter might change into a single address again. For
example, one could think of connecting to one server and retrieve the
list of server addresses from it.


\begin{lstlisting}
skv_status_t Disconnect();
\end{lstlisting}

Disconnecting from server(s).  Notice, the lack of distinction between
different server connections, means that one client object can have a
connection to only one server group at a time.  If an application
needs to connect to different servers, it has to use multiple client
objects that are initialized differently.



\subsection{Partitioned Data Set Operations}\label{sec:api:pds_ops}

\begin{lstlisting}
skv_status_t Open( char* aPDSName,
                   skv_pds_priv_t aPrivs,
                   skv_cmd_open_flags_t aFlags,
                   skv_pds_id_t* aPDSId );

skv_status_t iOpen( char* aPDSName,
                    skv_pds_priv_t aPrivs,
                    skv_cmd_open_flags_t aFlags,
                    skv_pds_id_t* aPDSId,
                    skv_client_cmd_ext_hdl_t* aCmdHdl );
\end{lstlisting}

Open access to a Partitioned Data Set (PDS).  There's a blocking and a
non-blocking version of the Open() call.
\begin{description}
\item[aPDSName] A \code{char*} representation of the PDS.  Each PDS
  needs to have a name represented by a c-style string.

\item[aPrivs] The privileges that will be used while the PDS is open.
  \begin{description}
  \item[\tiny SKV\_PDS\_READ] Allow users to read data from PDS.
  \item[\tiny SKV\_PDS\_WRITE] Allow users to write data to the PDS.
  \end{description}

\item[aFlags] A set of flags determining the response of the server
  under certain conditions.
  \begin{description}
  \item [\tiny SKV\_COMMAND\_OPEN\_FLAGS\_NONE] No special flags. This is
    the default.
  \item[\tiny SKV\_COMMAND\_OPEN\_FLAGS\_CREATE] Create a new PDS if it
    doesn't exists.
  \item[\tiny SKV\_COMMAND\_OPEN\_FLAGS\_EXCLUSIVE] Only try to open the PDS
    if it exists.  Otherwise return an error.
  \item[\tiny SKV\_COMMAND\_OPEN\_FLAGS\_DUP] Allow opening of a PDS that
    has duplicates. (Not fully supported at the moment)
  \end{description}

\item[aPDSId] Open will return a PDSId which is required for any
  further interaction with this PDS.  Inserting or retrieving data,
  everything needs the returned PDSId.

\item[aCmdHdl] If the non-blocking version of the command is used,
  this will return the command handle to check for completion of the
  operation.
\end{description}


\begin{lstlisting}
skv_status_t PDScntl( skv_pdscntl_cmd_t aCmd,
                      skv_pds_attr_t *aPDSAttr );

skv_status_t iPDScntl( skv_pdscntl_cmd_t aCmd,
                       skv_pds_attr_t *aPDSAttr,
                       skv_client_cmd_ext_hdl_t *aCmdHdl );
\end{lstlisting}

Perform various commands for a PDS.  This allows to retrieve
information about an existing PDS.  For example to return the
attributes.
\begin{description}
\item[aCmd] The command to execute.
  \begin{description}
    \item[\tiny SKV\_PDSCNTL\_CMD\_STAT\_SET] Set the attributes of a
      PDS.
    \item[\tiny SKV\_PDSCNTL\_CMD\_STAT\_GET] Get the attributes of a
      PDS.
    \item[\tiny SKV\_PDSCNTL\_CMD\_CLOSE] Close a PDS.  There's a
      separate Close() call available which has a simpler interface.
  \end{description}

\item[aPDSAttr] The attribute argument serves as both, input and
  output parameter.  The input has at least to be filled with the PDS
  name or the PDSid in order to find the PDS for the command.

\item[aCmdHdl] If the non-blocking version of the command is used,
  this will return the command handle to check for completion of the
  operation.
\end{description}


\begin{lstlisting}
skv_status_t Close( skv_pds_id_t* aPDSId );

skv_status_t iClose( skv_pds_id_t* aPDSId,
                     skv_client_cmd_ext_hdl_t* aCmdHdl );
\end{lstlisting}

Close an open PDS with a given PDSId.
\begin{description}
\item[aPDSId] The id of the PDS to close.  This id is returned by Open().
\item[aCmdHdl] If the non-blocking version of the command is used,
  this will return the command handle to check for completion of the
  operation.
\end{description}




\subsection{Inserting KV-Pairs}\label{sec:api:insert}

\begin{lstlisting}
skv_status_t Insert( skv_pds_id_t* aPDSId,
                     char* aKeyBuffer,
                     int aKeyBufferSize,
                     char* aValueBuffer,
                     int aValueBufferSize,
                     int aValueBufferOffset,
                     skv_cmd_RIU_flags_t aFlags );

skv_status_t iInsert( skv_pds_id_t* aPDSId,
                      char* aKeyBuffer,
                      int aKeyBufferSize,
                      char* aValueBuffer,
                      int aValueBufferSize,
                      int aValueBufferOffset,
                      skv_cmd_RIU_flags_t aFlags,
                      skv_client_cmd_ext_hdl_t* aCmdHdl );
\end{lstlisting}

Inserts a key-value pair to a PDS.
\begin{description}
\item[aPDSId] The PDSId returned from the Open() call.
\item[aKeyBuffer] Pointing to a buffer that contains the key.
\item[aKeyBufferSize] The size of the key.  Note, the maximum size of
  the key is currently limited by the size of the control message
  (\code{SKV\_CONTROL\_MESSAGE\_SIZE} defined in file
  include/common/skv\_types.hpp.
\item[aValueBuffer] Pointing to the buffer that contains the value.
\item[aValueBufferSize] The size of the value to insert.
\item[aValueBufferOffset] The offset at which the value buffer will be
  inserted at the server.  The offset has no effect on the source
  value buffer.
\item[aFlags] Insert flags described in Section\,\ref{sec:api:RIU:flags}.
\item[aCmdHdl] If the non-blocking version of the command is used,
  this will return the command handle to check for completion of the
  operation.
\end{description}


\subsection{Update KV-Pairs}\label{sec:api:update}
\begin{lstlisting}
skv_status_t Update( skv_pds_id_t* aPDSId,
                     char* aKeyBuffer,
                     int aKeyBufferSize,
                     char* aValueBuffer,
                     int aValueUpdateSize,
                     int aOffset,
                     skv_cmd_RIU_flags_t aFlags );

skv_status_t iUpdate( skv_pds_id_t* aPDSId,
                      char* aKeyBuffer,
                      int aKeyBufferSize,
                      char* aValueBuffer,
                      int aValueUpdateSize,
                      int aOffset,
                      skv_cmd_RIU_flags_t aFlags,
                      skv_client_cmd_ext_hdl_t* aCmdHdl );
\end{lstlisting}
This command is new and not yet implemented at the server. The
interface might change in the near future to include update triggers.


\subsection{Retrieve KV-Pairs}\label{sec:api:retrieve}
\begin{lstlisting}
skv_status_t Retrieve( skv_pds_id_t* aPDSId,
                       char* aKeyBuffer,
                       int aKeyBufferSize,
                       char* aValueBuffer,
                       int aValueBufferSize,
                       int* aValueRetrievedSize,
                       int aOffset,
                       skv_cmd_RIU_flags_t aFlags );

skv_status_t iRetrieve( skv_pds_id_t* aPDSId,
                        char* aKeyBuffer,
                        int aKeyBufferSize,
                        char* aValueBuffer,
                        int aValueBufferSize,
                        int* aValueRetrievedSize,
                        int aOffset,
                        skv_cmd_RIU_flags_t aFlags,
                        skv_client_cmd_ext_hdl_t* aCmdHdl );
\end{lstlisting}
Retrieves the value (or parts) for a given key.
\begin{description}
\item[aPDSId] The PDSId returned from the Open() call.
\item[aKeyBuffer] Pointing to a buffer that contains the key.
\item[aKeyBufferSize] The size of the key.  Note, the maximum size of
  the key is currently limited by the size of the control message
  (\code{SKV\_CONTROL\_MESSAGE\_SIZE} defined in file
  include/common/skv\_types.hpp.
\item[aValueBuffer] Pointing to the buffer that will hold the value
  after completion.
\item[aValueBufferSize] The number of bytes that the user requests to
  retrieve from the server.  Needs to be less than or equal to the
  size of the ValueBuffer.
\item[aValueRetrievedSize] Returns the actual number of retrieved
  bytes if the parameter is not NULL and the return status is
  \code{SKV\_SUCCESS}
\item[aOffset] The offset in the stored value where the retrieve
  should start to fetch data into \code{aValueBuffer}.
\item[aFlags] Retrieve flags. (See Section\,\ref{sec:api:RIU:flags})
\item[aCmdHdl] If the non-blocking version of the command is used,
  this will return the command handle to check for completion of the
  operation.
\end{description}


% \paragraph{Retrieve Size and Return Code}
% The returned data size and error code maybe counter-intuitive.
% However, this way it allows the retrieve command to return more
% information about the stored value at the server.  This extended
% semantics of parameter \code{ValueRetrievedSize} will return
% \code{SKV\_ERRNO\_VALUE\_TOO\_LARGE} if the data at the server is
% larger than the requested size.  In this situation, the data is
% properly retrieved into the user buffer and the retrieved size matches
% the requested size.  It's only that the return value in
% \code{ValueRetrievedSize} reflects the total size of the stored value
% and therefore will be larger than \code{ValueBufferSize}.

\subsection{Flags for Insert, Retrieve, and Update}\label{sec:api:RIU:flags}
Explanation of insert flags:  \todo{ls: need check for implementation
  inconsistencies of some flags}
\begin{description}
\item[SKV\_COMMAND\_RIU\_FLAGS\_NONE] This is the default.  If a
  record with the same key already exists, SKV returns an error
  (\code{SKV\_ERRNO\_RECORD\_ALREADY\_EXISTS}).
\item[SKV\_COMMAND\_RIU\_APPEND] If a record with the same key
  already exists, the new value is appended at the end of the existing
  value(s).
\item[SKV\_COMMAND\_RIU\_UPDATE] If a record with the same key already
  exists, the existing value is overwritten with the new one.  If the
  new value has a different size, the size is adjusted to match the
  lenght of the new value.  This also includes partial updates with
  provided offsets as long as $offset <= existing\_value\_size$.
\item[SKV\_COMMAND\_RIU\_INSERT\_EXPANDS\_VALUE] When used with the
  proper offset, this flag can be used to append new data at or after
  the end of existing values.  The semantics is the same as with
  append except that the user is responsible for using correct
  offsets.  Note that the provided offset has to be larger or equal to
  the size of the existing value (non-overlapping inserts).
\item[SKV\_COMMAND\_RIU\_INSERT\_OVERWRITE\_VALUE\_ON\_DUP] Overwrites
  existing parts of the value without adjusting the size of the
  record.  Only works properly if $offset + length <= size$.
\item[SKV\_COMMAND\_RIU\_INSERT\_OVERLAPPING] Allow overlapping
  updates. For example in conjunction with
  \code{SKV\_COMMAND\_RIU\_INSERT\_EXPANDS\_VALUE} to enable arbitrary
  partial updates to a value.
\item[SKV\_COMMAND\_RIU\_RETRIEVE\_SPECIFIC\_VALUE\_LEN] Enable report of
  the totalsize even if less data was requested.
\end{description}


\subsection{Remove KV-Pairs}\label{sec:api:remove}
\begin{lstlisting}
skv_status_t Remove( skv_pds_id_t* aPDSId,
                     char* aKeyBuffer,
                     int aKeyBufferSize,
                     skv_cmd_remove_flags_t aFlags );

skv_status_t iRemove( skv_pds_id_t* aPDSId,
                      char* aKeyBuffer,
                      int aKeyBufferSize,
                      skv_cmd_remove_flags_t aFlags,
                      skv_client_cmd_ext_hdl_t* aCmdHdl );
\end{lstlisting}
Deletes a Key/Value pair from the server storage.
\begin{description}
\item[aPDSId] The PDSId returned from the Open() call.
\item[aKeyBuffer] Pointing to a buffer that contains the key.
\item[aKeyBufferSize] The size of the key.  Note, the maximum size of
  the key is currently limited by the size of the control message
  (\code{SKV\_CONTROL\_MESSAGE\_SIZE} defined in file
  include/common/skv\_types.hpp.
\item[aFlags] Command flags for remove. \todo{check if used at all}
\item[aCmdHdl] If the non-blocking version of the command is used,
  this will return the command handle to check for completion of the
  operation.
\end{description}


\subsection{Bulk Insert}\label{sec:api:bulkinsert}
Allows accumulated insertion of data.

\begin{lstlisting}
skv_status_t CreateBulkInserter( skv_pds_id_t* aPDSId,
                                 skv_bulk_inserter_flags_t aFlags,
                                 skv_client_bulk_inserter_ext_hdl_t* aBulkInserterHandle );
\end{lstlisting}
Creates a context/handle for bulk insertion of data.
\begin{description}
\item[aPDSId] The PDSId returned from the Open() call.
\item[aFlags] Bulk-inserter flags for configuration.
\item[aBulkInserterHandle] returned context/handle for use with
  insert, flush, and close operations.
\end{description}

\begin{lstlisting}
skv_status_t Insert( skv_client_bulk_inserter_ext_hdl_t aBulkInserterHandle,
                     char* aKeyBuffer,
                     int aKeyBufferSize,
                     char* aValueBuffer,
                     int aValueBufferSize,
                     skv_bulk_inserter_flags_t aFlags
                     );
\end{lstlisting}
Add a key/value pair into the bulk-inserter.  If the internal buffer
reaches a threshold, it will be automatically flushed. Otherwise, the
Flush() command can be used.
\begin{description}
\item[aBulkInserterHandle] The context/handle returned from the
  CreateBulkInserter() call.
\item[aKeyBuffer] Pointing to a buffer that contains the key.
\item[aKeyBufferSize] The size of the key.  Note, the maximum size of
  the key is currently limited by the size of the control message
  (\code{SKV\_CONTROL\_MESSAGE\_SIZE} defined in file
  include/common/skv\_types.hpp.
\item[aValueBuffer] Pointing to the buffer that will hold the value
  after completion.
\item[aValueBufferSize] The number of bytes that the user requests to
  retrieve from the server.  Needs to be less than or equal to the
  size of the ValueBuffer.
\item[aFlags] Only \code{SKV\_BULK\_INSERTER\_FLAGS\_NONE} is
  defined yet.
\end{description}

\begin{lstlisting}
skv_status_t Flush( skv_client_bulk_inserter_ext_hdl_t aBulkInserterHandle );
\end{lstlisting}
\begin{lstlisting}
skv_status_t CloseBulkInserter( skv_client_bulk_inserter_ext_hdl_t aBulkInserterHandle );
\end{lstlisting}

  
\subsection{Asynchronous Completion}\label{sec:api:async}
\begin{lstlisting}
skv_status_t TestAny( skv_client_cmd_ext_hdl_t* aCmdHdl );

skv_status_t Test( skv_client_cmd_ext_hdl_t aCmdHdl );
\end{lstlisting}
Non-blocking test if a specific \code{aCmdHdl} is complete or not.
\code{TestAny} will return a valid handle if any outstanding request
is complete.

Once a test is successful, there's no further need for cleanup by the
application.  That means, a particular \code{aCmdHdl} will only show
up once whether the application uses test or wait variants of the
calls.

\begin{lstlisting}
skv_status_t WaitAny( skv_client_cmd_ext_hdl_t* aCmdHdl );

skv_status_t Wait( skv_client_cmd_ext_hdl_t aCmdHdl );
\end{lstlisting}
Blocking test if a specific \code{aCmdHdl} is complete or not.  The
call blocks until either any (\code{WaitAny()}) or a particular
(\code{Wait()}) command completed.




\subsection{Iterators/Cursors}\label{sec:api:cursor}
Cursors enable iteration over the existing key/value pairs of a given
PDS.  There are 2 flavors of iterators\footnote{The terms cursor and
  iterator will be used as synonyms here.}: a local and a global
cursor.

The main mode of operation for both flavors of the cursors:
\begin{enumerate}
\item Open a cursor to get a handle for further operation
\item Call \verb|GetFirst(Local)Element()| to seek to the beginning of
  the list of entries.
\item Call \verb|GetNext(Local)Element()| to iterate
\item Close the cursor to release resources and invalidate the handle.
\end{enumerate}


\subsubsection{Local Iterator}\label{sec:api:cursor_local}
The local cursor allows iteration over the key/value entries of only
one server.  The term \emph{local} might be misleading because the
client library will contact the server with a given MPI rank.  This
may be non-intuitive especially if servers and clients run on disjoint
sets of nodes/machines.

\begin{lstlisting}
skv_status_t OpenLocalCursor( int aNodeId,
                              skv_pds_id_t* aPDSId,
                              skv_client_cursor_ext_hdl_t* aCursorHdl );
\end{lstlisting}
\begin{description}
\item[aNodeId] Provide the NodeID or MPI-rank of the server you want
  to iterate.
\item[aPDSId] The PDSId returned from the Open() call.
\item[aCursorHdl] This output parameter will return a handle for the
  iterator to be used for the \verb|GetLocalNNN()| calls.
\end{description}

\begin{lstlisting}
skv_status_t GetFirstLocalElement( skv_client_cursor_ext_hdl_t aCursorHdl,
                                   char* aRetrievedKeyBuffer,
                                   int* aRetrievedKeySize,
                                   int aRetrievedKeyMaxSize,
                                   char* aRetrievedValueBuffer,
                                   int* aRetrievedValueSize,
                                   int aRetrievedValueMaxSize,
                                   skv_cursor_flags_t aFlags );

skv_status_t GetNextLocalElement( skv_client_cursor_ext_hdl_t aCursorHdl,
                                  char* aRetrievedKeyBuffer,
                                  int* aRetrievedKeySize,
                                  int aRetrievedKeyMaxSize,
                                  char* aRetrievedValueBuffer,
                                  int* aRetrievedValueSize,
                                  int aRetrievedValueMaxSize,
                                  skv_cursor_flags_t aFlags );
\end{lstlisting}

\begin{description}
\item[aCursorHdl] The cursor handle returned from the
  OpenLocalCursor() call.
\item[aRetrievedKeyBuffer] Points to a buffer that will contain the
  first retreived key.
\item[aRetrievedKeySize] Will contain the size of the retrieved key
  when the call returns.
\item[aRetrievedKeyMaxSize] The maximum size of any key expected for
  the cursor (\abrIE the size of the provided buffer).
\item[aRetrievedValueBuffer] Points to a buffer that will contain the
  first retrieved value.
\item[aRetrievedValueSize] Will contain the size of the retrieved
  value when the call returns.
\item[aRetrievedValueMaxSize] The maximum size of any value expected
  for the cursor (\abrIE the size of the provided buffer).
\item[aFlags] Currently not in use.  First- and Next- call set these
  flags accordingly so that there's nothing to do for the user.
\end{description}

\begin{lstlisting}
skv_status_t CloseLocalCursor( skv_client_cursor_ext_hdl_t aCursorHdl );
\end{lstlisting}

Closes the cursor and invalidates the cursor handle.
\begin{description}
\item[aCursorHdl] The cursor handle returned from the
  OpenLocalCursor() call.
\end{description}

\subsubsection{Global Iterator}\label{sec:api:cursor_global}
The global iterator will iterate over all available key/value pairs of
a PDS on all connected servers.  The usage of the API calls is the
same as for the local generator.

\begin{lstlisting}
skv_status_t OpenCursor( skv_pds_id_t* aPDSId,
                         skv_client_cursor_ext_hdl_t* aCursorHdl );

skv_status_t CloseCursor( skv_client_cursor_ext_hdl_t aCursorHdl );

skv_status_t GetFirstElement( skv_client_cursor_ext_hdl_t aCursorHdl,
                              char* aRetrievedKeyBuffer,
                              int* aRetrievedKeySize,
                              int aRetrievedKeyMaxSize,
                              char* aRetrievedValueBuffer,
                              int* aRetrievedValueSize,
                              int aRetrievedValueMaxSize,
                              skv_cursor_flags_t aFlags );

skv_status_t GetNextElement( skv_client_cursor_ext_hdl_t aCursorHdl,
                             char* aRetrievedKeyBuffer,
                             int* aRetrievedKeySize,
                             int aRetrievedKeyMaxSize,
                             char* aRetrievedValueBuffer,
                             int* aRetrievedValueSize,
                             int aRetrievedValueMaxSize,
                             skv_cursor_flags_t aFlags );
\end{lstlisting}


\subsection{Example Client}\label{sec:api:example}
Source not pasted here.  There are several examples that can be found
in the SKV source tree under \verb|test|.

\endinput



%%% Local Variables: 
%%% mode: latex
%%% TeX-master: "skvdoc"
%%% End: 
