%%%%%%%%%%%%%%%%%%%%%%%%%%%%%%%%%%%%%%%%
% Copyright (c) IBM Corp. 2014
% All rights reserved. This program and the accompanying materials
% are made available under the terms of the Eclipse Public License v1.0
% which accompanies this distribution, and is available at
% http://www.eclipse.org/legal/epl-v10.html
%
% Contributors:
%    lschneid
%%%%%%%%%%%%%%%%%%%%%%%%%%%%%%%%%%%%%%%%

\section{Command Paths}\label{sec:skv:commands}

\subsection{Init}
\code{Init()} is the first routine to call when creating a SKV
client.
\begin{enumerate} \parskip-0.5ex
\item \code{skv\_client\_t::Init()}
\item \code{skv\_client\_internal\_t::Init()}
  \begin{enumerate} \parskip-0.5ex
  \item creates IT\_API device
  \item creates IT\_API protection zone
  \item initialize Command Control Block Manager Interface
  \item initialize Connection Manager Interface
  \item initialize Command Manager Interface
  \item initialize Relational Cursor Manager Interface
  \end{enumerate}
\end{enumerate}


\subsection{Connect}
Connect to a SKV server.
\begin{enumerate} \parskip-0.5ex
\item \code{skv\_client\_t::Connect()}
\item \code{skv\_client\_internal\_t::Connect()}
  \begin{enumerate} \parskip-0.5ex
  \item \code{skv\_client\_conn\_manager\_if\_t::Connect()}
    \begin{enumerate} \parskip-0.5ex
    \item gets client hostname via \code{gethostname} to find out
      where it is running (determine method to get server IP)
    \item Get \code{ComputeFileNamePath} with server IPs
    \item Number of SKV servers determined by number of lines in\\ \code{ComputeFileNamePath}
    \item Fetches server addresses/names from file
    \item Chose a random server to connect first, then connect
      subsequent servers (\code{ConnectToServer()})
      \begin{enumerate}
      \item creates RC endpoint
      \item gets IPv4 address via \code{gethostbyname()}
      \item does \code{it\_ep\_connect()}
      \item polls the event dispatchers for connection manager msgs
        (and affiliated events and unaffiliated events for errors)
      \end{enumerate}
    \item Retrieves the distribution function
      \begin{enumerate}
      \item Creates a command

        calls \code{skv\_client\_command\_manager\_if\_t::Reserve()}
        which removes an element from queue of free Command Control
        Blocks (CCB). The queue is implemented as a double linked
        list and implements a stack semantic (LIFO).
      \item initializes a \code{skv\_cmd\_retrieve\_dist\_req\_t}
      \item dispatch the command to a random server
      \item wait for the command to complete
      \end{enumerate}
    \end{enumerate}
  \end{enumerate}
\end{enumerate}


\subsection{Open}
Create or open a partitioned data set (PDS).  Open uses \code{iOpen} and \code{wait}.
\begin{itemize}
\item Use the PDS-name as a key and decide the server number of the owner
\end{itemize}




\section{The SKV Server State Machine}\label{sec:skv:srv_sm}

Questions to answer in this section:
\begin{itemize}
\item Which server states are defined
\item Event types that cause which transition
\item Which events are triggered when transitioning
\end{itemize}


\section{Implementation}\label{sec:skv:impl}

\subsection{Endpoint State}
An Endpoing State (EPState) holds connection-relevant information and
keeps track of QP resources.

\emph{Architecturally this might be questionable. A QP status is
  required to maintain. However, since the main server state machine
  multiplexes all events through a single aggregated event queue, it
  requires another demultiplexing step to get into the EPStates and
  hold their information... Maybe this is the best solution already. }



\subsection{Command Control Block}
A command control block (CCB) is assigned to each command/request that is
\emph{currently processed} or \emph{pending}. It contains the required
information about the command and its status.
An essential attribute is the \emph{command ordinal}.

There's a stack of available CCBs for each EPState.  The following
steps describe the CCB handling:
\begin{enumerate}
\item A free CCB is assigned to an event at the time of
  recv-completion for a new request.  This happens before the
  initialization of the event (routine \code{GetITEvent()}).
\item The command ordinal of the new CCB is assigned to the event to
  be able to find the correct CCB later.
\item Processing of the command
\item The CCB is returned to the pile of free slots after successfull
  dispatching of the response to the client.  Note: since the involved
  send and recv buffers are detached from the CCB, this return already
  happens after send and recv operations are posted regardless of
  their completion status.  For send and recv buffers see below.
\end{enumerate}


\subsection{Command Send Buffers}
These buffers hold the data that is sent back to the client.  They
have to be preserved until the send completion event is processed and
therefore are detached from the handling of the CCB.

An available send buffer is always attached to a single CCB.  The send
buffer is replaced by a spare after the response to the client is
posted and before the CCB is returned to the free command slots list.
Replacing the send buffer means to get an unposted buffer from a stack
of buffers.

A posted buffer is returned to the stack if the corresponding send
operation is completed (send completion event).



\subsection{Command Recv Buffers}
Recv buffers provide space to receive requests from a client.  They
have been detached from the CCB because of a potential race
condition.

As a new request arrives (in a receive buffer), this buffer gets
assigned/attached to a free CCB.  This assignment is kept until the
command is processed and the response is dispatched.  Right before
posting a new recv for future commands, the old recv buffer is
detached from the CCB and 


\subsection{Client-Server Protocol}

Header data consists of the command type (Insert, Retrieve, Remove,
...), a command ordinal, and the address of a command control block.
The command ordinal and the CCB information is primarily required by
the client to pick the right client CCB after receiving the response
from the server.


At the server, the header data is generally kept in the receive buffer
that is attached to the CCB.  An exception is if the command requires
2 phases, \abrIE RDMA-read in case of the insert command.  The
header information is copied from the receive buffer to the
CommandState member of the CCB for later use.




\endinput



%%% Local Variables: 
%%% mode: latex
%%% TeX-master: "skvdoc"
%%% End: 
